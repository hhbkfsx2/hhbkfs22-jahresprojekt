\documentclass[10pt,a4paper,oneside,ngerman]{article}

\usepackage[left=2.5cm,right=2.5cm,top=2.5cm,bottom=2.5cm]{geometry}

\usepackage[utf8]{inputenc}					% this is needed for german umlauts
\usepackage[ngerman]{babel}					% this is needed for german umlauts
\usepackage[T1]{fontenc}						% this is needed for correct output 
                           					% of umlauts in pdf

\usepackage{lmodern}							% Latin Modern
\renewcommand*\familydefault{\sfdefault}		% Only if the base font of the document is to be sans serif
\usepackage[T1]{fontenc}

%%% Anfang: Anpassen des Layouts: Kopf- und Fußzeilen %%%

\usepackage{fancyhdr} 
\pagestyle{fancy} 
\fancyhf{} 									% alle Kopf- und Fußzeilenfelder bereinigen; 
											% löscht doppelte Seitenzahlen!
\fancyhead[L]{} 								% Kopfzeile links
\fancyhead[C]{}								% zentrierte Kopfzeile
\fancyhead[R]{} 								% Kopfzeile rechts
%\fancyhead[LO]{Test}						% Odd: ungerade Seiten
%\fancyhead[RO]{Test}
%\fancyhead[CO]{Test}
%\fancyhead[RE]{Test}						% Even: gerade Seiten
%\fancyhead[LE]{Test}
%\fancyhead[CE]{Test}
\fancyfoot[R]{\thepage} %Seitennummer

%\renewcommand{\headrulewidth}{0.4pt} %obere Trennlinie
%\renewcommand{\footrulewidth}{0.4pt} %untere Trennlinie

%%% Ende: Anpassen des Layouts: Kopf- und Fußzeilen %%%

\usepackage{graphicx}

\usepackage{color}							% Ermöglicht farbigen Text: \textcolor{declared-color}{text}
											% Bsp.: \textcolor{red}{Dieser Text ist rot}.

%%% Zeilenabstand %%%
\usepackage{setspace}
\onehalfspacing %auskommentiert wird durchgehend einfacher Zeilenabstand verwendet

\newcommand{\qr}{\textquotedblleft}		% Definiert eine Abkürzung für dt. Anführungsstriche links
\newcommand{\ql}{\quotedblbase}			% Definiert eine Abkürzung für dt. Anführungsstriche rechts

\newcommand{\tabb}{\begin{tabbing}
m \= m \= m \= m \= m \= m \= m \= m \= m \= m \= m \kill}
\newcommand {\tabx}{\end{tabbing}}

\usepackage{booktabs}
\usepackage{tabularx}
\usepackage{listings}

%%% Nuetzliche Befehle:
%Behauptung \footnote{Beweis}; wird automatisch durchnummeriert 

%%% Anfang: Generelle Inforationen über das Dokument %%%

\author{Kilian Engelhardt}
\title{}
\date{\today}

\usepackage{hyperref}					% Weitere Optionen unter:
										% http://de.wikibooks.org/wiki/LaTeX-W%C3%B6rterbuch:_hyperref
\hypersetup{								% Hier werden Informationen für das PDF gesetzt
pdftitle={},
pdfauthor={},
pdfsubject={},
pdfkeywords={},
colorlinks=true,							
linkcolor=black							% Definiert die Farbe der Links vom Inhaltsverzeichnis
}										% zu den Sections im Dokument

\usepackage{wrapfig}

%%% Ende: Generelle Inforationen über das Dokument %%%

%\setcounter{secnumdepth}{0} 				% Entfernt Nummerierung im TOC

\begin{document}
\pagestyle{empty}
\include{8gruppe_projektstruktur_00_deckblatt}
\tableofcontents
\newpage
\pagestyle{fancy}
\setcounter{page}{1}

\section{Projektstrukturplan}

Liste der geplanten Arbeitspackete und der jeweiligen Anzahl von geplanten Stunden.\\
\noindent \begin{tabular}{|l|l|l|}
\hline
Name & Beschreibung & Stunden \\
AP001 & Zusammentragen und Überprüfen der Hardware & 1h\\
AP002 & Router: Basis-Konfiguration (IPv6) & 4h\\
AP003 & Switch: Basis-Konfiguration (IPv6) & 4h\\
AP004 & Router: Routing Intern & 4h\\
AP005 & Skript: Netzwerk Tests & 4h\\
AP006 & Test: Netzverfügbarkeit \& Konnektivität & 1h\\
AP007 & Test: Netzwerk Reinheit (kein IPv4 / Wireshark) & 1h\\
AP008 & Router: SIXXS-Tunnel & 4h\\
AP009 & Test: Tunnel Stabilität/Konnektivität/Goodwidth & 1h\\
AP010 & Router IPv6 Firewall & 4h\\
AP011 & Backup: Router \& Switch Konfiguration & 1h\\
AP012 & Installation: Hypervisor & 2h\\
AP013 & Konfiguration Hypervisor & 4h\\
AP014 & Installation: Linux Server mit LAMP & 5h\\
AP015 & Konfiguation: Web-/ Mailserver & 6h\\
AP016 & Test: Webserver intern & 1h\\
AP017 & Test: Webserver extern & 1h\\
AP018 & Test: Mailversand intern & 1h\\
AP019 & Test: Mailversand extern & 1h\\
AP020 & Test: Firewall & 1h\\
AP021 & Installation: Windows Domain Controller & 2h\\
AP022 & Konfiguration: Windows Domain Controller & 6h\\
AP023 & Test: Domänen Verfügbarkeit & 1h\\
AP024 & Test: Authentizierung am AD & 1h\\
AP102 & Dokumentation: Konfiguration Router & 3h\\
AP103 & Dokumentation: Konfiguration Switch & 3h\\
AP104 & Dokumentation: Skripte & 3h\\
AP105 & Dokumentation: Test Netzwerk Intern & 2h\\
AP106 & Dokumentation: Test Netzwerk Extern & 2h\\
AP107 & Dokumentation: Hypervisor & 3h\\
AP108 & Dokumentation: Linux Server LAMP & 3h\\
AP109 & Dokumentation: Linux Server Mail & 3h\\
AP110 & Dokumentation: Test Linux Server & 2h\\
AP111 & Dokumentation: Test Firewall & 2h\\
AP112 & Dokumentation: Domaincontroller & 3h\\
AP113 & Dokumentation: Test Domaincontroller & 2h\\

\hline
\end{tabular}
\vspace{0.5cm}\\
Liste der geplanten Zuordnung der Arbeitspackete\\
\noindent \begin{tabular}{|l|c|}
\hline
Tom Vogler & AP001, AP003, AP007, AP011, AP009, AP016, AP017, AP021, AP022, AP112, AP113 \\
Kilian Engelhardt & AP001, AP005, AP006, AP012, AP013, AP014, AP015, AP023, AP024, AP107, AP108, AP109, AP110 \\
Mirko Grossmann & AP001, AP002, AP004, AP008, AP010, AP018, AP019, AP020, AP005, AP102, AP103, AP104, AP105, AP106 \\
\hline
\end{tabular}

Die Arbeitspakete AP007, AP009, AP013, AP017, AP019 und AP024 stellen Meilensteine dar. Dabei handelt es sich jeweils um den Zeitpunkt, zudem die entsprechenden Komponenten konfiguriert und getestet wurden.

 


\end{document}

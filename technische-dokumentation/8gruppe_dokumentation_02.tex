\section{Router: Konfiguration}

Abweichend von der einleitenden Anmerkung wurden folgende Befehle unter Ciscos iOS verwendet, um die Konfiguration des Routers vorzunehmen.

\begin{lstlisting}[numbers=none]
#Basics
r1(config)#enable secret password123
r1(config)#enable password password123
r1(config)#ipv6 unicast-routing
r1(config)#ip name-server 8.8.8.8

#Vlan Deklaration
r1#vlan database 
r1(vlan)#vlan 10
r1(vlan)#vlan 20
r1(vlan)#apply
r1(vlan)#exit

#Subinterface vlan 10
r1(config)#interface FastEthernet0/1.10
r1(config-subif)#description subinterface vlan 10
r1(config-subif)#encapsulation dot1Q 10
r1(config-subif)#ipv6 address 2001:4dd0:fc0b:a::1/64
r1(config-subif)#no shutdown
r1(config-subif)#exit

#Subinterface vlan 20
r1(config)#interface FastEthernet0/1.20
r1(config-subif)#description subinterface vlan 20
r1(config-subif)#encapsulation dot1Q 20 native
r1(config-subif)#ipv6 address 2001:4dd0:fc0b:f4::1/64
r1(config-subif)#no shutdown
r1(config-subif)#exit

#Interface ins Schulnetz
r1(config)#interface fastethernet 0/0
r1(config-if)#ip address 212.72.180.241 255.255.255.224
r1(config-if)#ip default-gateway 212.72.180.225
r1(config-if)#no shutdown
r1(config-if)#exit

#SSH
r1(config)#ip domain-name fastforward.hhbk.de
r1(config)#crypto key generate rsa general-keys modulus 1024
r1(config)#username admin privilege 15 secret password123
r1(config)#line vty 0 4
r1(config-line)#transport input telnet ssh
r1(config-line)#login local
r1(config-line)#end

#Routing
r1(config)#ip route 0.0.0.0 0.0.0.0 fastethernet 0/0
r1(config)#ipv6 route 2001:4dd0:fc0b:a::/64 FastEthernet0/1.10
r1(config)#ipv6 route 2001:4dd0:fc0b:f4::/64 FastEthernet0/1.20

#SixXS Tunnel
r1(config)#interface Tunnel61
r1(config-if)#description 6in4 tunnel to SixXS
r1(config-if)#no ip address
r1(config-if)#ip tcp adjust-mss 1420
r1(config-if)#ipv6 address 2001:4dd0:ff00:147f::2/64
r1(config-if)#ipv6 enable
r1(config-if)#tunnel source fastethernet 0/0
r1(config-if)#tunnel destination 78.35.24.124
r1(config-if)#tunnel mode ipv6ip
r1(config-if)#exit
r1(config)#ipv6 route ::/0 Tunnel61

#Tunnel Prüfen
r1(config)#do show ip interface tunnel61
r1(config)#do show ipv6 interface tunnel61
\end{lstlisting}

Konfiguration der Firewall:
\begin{lstlisting}[numbers=none]
#Firewalling

#WAN In

r1(config)#ipv6 access-list from_wan_in
r1(config-ipv6-acl)#permit icmp any any
r1(config-ipv6-acl)#permit tcp any any eq 22 reflect dmz-wan-reflexive timeout 300
r1(config-ipv6-acl)#permit tcp any any eq www reflect dmz-wan-reflexive timeout 5
r1(config-ipv6-acl)#permit tcp any any eq 443 reflect dmz-wan-reflexive timeout 5
r1(config-ipv6-acl)#permit tcp any any eq smtp reflect dmz-wan-reflexive timeout 5
r1(config-ipv6-acl)#evaluate wan-dmz-reflexive
r1(config-ipv6-acl)#evaluate wan-lan-reflexive
r1(config-ipv6-acl)#exit

r1(config)#interface Tunnel61
r1(config-if)#ipv6 traffic-filter from_wan_in in


#DMZ In

r1(config)#ipv6 access-list dmz_in
r1(config-ipv6-acl)#permit icmp any any
r1(config-ipv6-acl)#permit udp any any eq domain reflect wan-dmz-reflexive timeout 5
r1(config-ipv6-acl)#permit tcp any any eq 22 reflect wan-dmz-reflexive timeout 5
r1(config-ipv6-acl)#permit tcp any any eq www reflect wan-dmz-reflexive timeout 5
r1(config-ipv6-acl)#permit tcp any any eq 443 reflect wan-dmz-reflexive timeout 5
r1(config-ipv6-acl)#permit tcp any any eq smtp reflect wan-dmz-reflexive timeout 5
r1(config-ipv6-acl)#evaluate dmz-wan-reflexive
r1(config-ipv6-acl)#evaluate wan-lan-reflexive
r1(config-ipv6-acl)#exit

r1(config)#interface FastEthernet0/1.10
r1(config-if)#ipv6 traffic-filter dmz_in in


#LAN In

r1(config)#ipv6 access-list lan_in
r1(config-ipv6-acl)#permit ipv6 any any reflect wan-lan-reflexive timeout 5
r1(config-ipv6-acl)#exit

r1(config)#interface FastEthernet0/1.20
r1(config-if)#ipv6 traffic-filter lan_in in


#IPv4 WAN In
r1(config)#ip access-list extended from_wan_v4_in
r1(config-ext-nacl)#permit ip host 78.35.24.124 any
r1(config-ext-nacl)#permit tcp any any eq 22
r1(config-ext-nacl)#exit

r1(config)#interface FastEthernet0/0
r1(config-if)#ip access-group from_wan_v4_in in
r1(config-ext-nacl)#permit icmp any any
r1(config-ext-nacl)#exit


#Firewall Regeln auslesen
r1(config)#do show ipv6 access-list
r1(config)#do show ip access-list
\end{lstlisting}

\section{Tests}

\subsection{Erreichbarkeit intern}

Mit dem folgenen Skript wurde die allgemeine Erreichbarkeit der Server aus dem LAN getestet.

Shell-Skript: {\sc test-ping.sh}
\begin{lstlisting}
#!/bin/bash

#killall dhclient

RIP1="2001:4dd0:fc0b:a::1"
RIP2="2001:4dd0:fc0b:f4::1"
SIP1="2001:4dd0:fc0b:a::2"
SIP2="2001:4dd0:fc0b:f4::2"
KVM1="2001:4dd0:fc0b:a::3"
KVM2="2001:4dd0:fc0b:f4::3"
SRV="2001:4dd0:fc0b:a::4"
DC="2001:4dd0:fc0b:f4::5"

LOG="test-ping_$(date +%Y%d%m).log"

IP="${RIP1} ${RIP2} ${SIP1} ${SIP2} ${KVM1} ${KVM2} ${SRV} ${DC}"

echo -e "##########" >> ${LOG}
echo -e "Ping-Test $(date +%Y%m%d):\n" >> ${LOG}
for i in ${IP}; do
	ping6 -c 1 ${i} 2> /dev/null
	if [[ $? -eq 0 ]]; then
		echo -e "${i}\t\tworks" >> ${LOG}
	else
		echo -e "${i}\t\tfailed!" >> ${LOG}
	fi
done	
echo -e "\n" >> ${LOG}
\end{lstlisting}

Log-Datei: {\sc test-ping\_20160622.log}
\begin{lstlisting}
##########
Ping-Test 20160622:

2001:4dd0:fc0b:a::1		works
2001:4dd0:fc0b:f4::1		works
2001:4dd0:fc0b:a::2		works
2001:4dd0:fc0b:f4::2		works
2001:4dd0:fc0b:a::3		works
2001:4dd0:fc0b:f4::3		works
2001:4dd0:fc0b:a::4		works
2001:4dd0:fc0b:f4::5		works
\end{lstlisting}

\subsection{Erreichbarkeit extern}

\subsubsection{Allgemeine Erreichbarkeit}

Zum Testen der allgemeinen Erreichbarkeit von extern wurde der Ping-Test an einem Internetanschluss mit Dualstack wiederholt.

\begin{lstlisting}
##########
Ping-Test 20160622:

2001:4dd0:fc0b:a::1		works
2001:4dd0:fc0b:f4::1		works
2001:4dd0:fc0b:a::2		works
2001:4dd0:fc0b:f4::2		works
2001:4dd0:fc0b:a::3		works
2001:4dd0:fc0b:f4::3		works
2001:4dd0:fc0b:a::4		works
2001:4dd0:fc0b:f4::5		works
\end{lstlisting}

\subsubsection{Erreichbarkeit Webserver}

Die Erreichbarkeit des Webservers wurde von der Kommandozeile per {\sc curl} getestet.

\begin{lstlisting}[numbers=none]
> curl -v fastforward.hhbk.de
 * Rebuilt URL to: fastforward.hhbk.de/
 * Hostname was NOT found in DNS cache
 *   Trying 2001:4dd0:fc0b:a::4...
 * Connected to fastforward.hhbk.de (2001:4dd0:fc0b:a::4) port 80 (#0)
 > GET / HTTP/1.1
 > User-Agent: curl/7.35.0
 > Host: fastforward.hhbk.de
 > Accept: */*
 > 
 < HTTP/1.1 200 OK
 < Date: Wed, 22 Jun 2016 15:03:00 GMT
 * Server Apache/2.4.18 (Ubuntu) is not blacklisted
 < Server: Apache/2.4.18 (Ubuntu)
 < Last-Modified: Mon, 20 Jun 2016 07:40:59 GMT
 < ETag: "d-535b0d3581c7e"
 < Accept-Ranges: bytes
 < Content-Length: 13
 < Content-Type: text/html
 < 
 Hello World!
 * Connection #0 to host fastforward.hhbk.de left intact
\end{lstlisting}

\subsubsection{Erreichbarkeit Mailserver}

Die Erreichbarkeit des Mailservers wurde von der Kommandozeile per {\sc telnet} getestet.

\begin{lstlisting}[numbers=none]
> telnet fastforward.hhbk.de 25
 Trying 2001:4dd0:fc0b:a::4...
 Connected to fastforward.hhbk.de.
 Escape character is '^]'.
 220 webserver ESMTP Postfix (Ubuntu)
 HELO fastforward.hhbk.de
 250 webserver
 mail from: <test@test.com>
 250 2.1.0 Ok
 rcpt to: <kilian@fastforward.hhbk.de>
 250 2.1.5 Ok
 subject: test
 Line one
 Line two
 .
 250 2.0.0 Ok: queued as 1BF1B6099B
 quit
 221 2.0.0 Bye
 Connection closed by foreign host.
\end{lstlisting}

\begin{lstlisting}[numbers=none]
> tail /var/mail/kilian 
 X-Original-To: kilian@fastforward.hhbk.de
 Delivered-To: kilian@fastforward.hhbk.de
 Received: from fastforward.hhbk.de (unknown [IPv6:2a02:908:1251:7160:495e:958d:9e35:f017])
	by webserver (Postfix) with SMTP id 1BF1B6099B
	for <kilian@fastforward.hhbk.de>; Wed, 22 Jun 2016 16:49:47 +0200 (CEST)

 subject: test
 Line one
 Line two
\end{lstlisting}

\subsection{Firewall}

Mit den folgenden Kommandos wurde getestet, ob die Firewall nicht freigeschaltete Ports blockiert. Dazu wurde auf dem Linux-Server mit {\sc netcat} ein Port geöffnet und von extern geprüft, ob sich zu diesem Port verbunden werden kann.

\begin{lstlisting}
 #Listening auf dem Linux-Server
> netcat -l -p 1337

 #Vom externen Host
> telnet 2001:4dd0:fc0b:a::4 1337
 Trying 2001:4dd0:fc0b:a::4...
 telnet: connect to address 2001:4dd0:fc0b:a::4: Permission denied
\end{lstlisting}

\section{Hypervisor}

\subsection{Installation}

Die Installation erfolgt per graphischen Installationsdialog. Englisch wurde gewählt, da es die Lingua franca in der IT darstellt. Zusätzlich wurde OpenSSH bei der Installation ausgewählt, um den Server ohne graphische Oberfläche aus der Ferne zu administrieren. Insgesamt wurden während der Installation folgende Einstellungen vorgenommen:
\begin{itemize}[leftmargin=+1in]
	\item[Language] Englisch
	\item[Territory] Germany
	\item[Keyboard] german
	\item[Hostname] hypervisor
	\item[Domain name] fastforward.hhbk.de
	\item[Username] user
	\item[Password] password123
	\item[Paritioning] Guided: use entire disk
	\item[Choose software] Default, OpenSSH
	\item[Grub MBR] sdb
\end{itemize}

Nach der Installation wurde darüberhinaus folgende Software installiert: {\texttt qemu-kvm libvirt-bin virtinst}.

\subsection{Konfiguration}

\subsubsection{libvirt}

Zunächst muss der QEMU-Treiber von {\texttt libvirt} konfiguriert werden, damit dieser weiß, mit welchem User QEMU ausgeführt wird.\newline

Konfigurationsdatei: {\texttt /etc/libvirt/qemu.conf}
\begin{lstlisting}
user = "root"
group = "root"
\end{lstlisting}

Anschließend wird die HDD mit 500GB formatiert und die Volume Group {\texttt vg0} definiert.\newline

\begin{lstlisting}[numbers=none]
> parted /dev/sda
   mklabel GPT
   mkpart primary 1M 100%
   set 1 lvm on
> pvcreate /dev/sda1
> vgcreate vg0 /dev/sda1
\end{lstlisting} 

Die Volume Group {\texttt vg0} wird verwendet, um den Pool {\texttt vg0} einzurichten. Die folgende Konfiguration muss erstellt werden, um anschließend mit den aufgeführten Befehlen den Pool zu aktivieren.

\newpage
Konfigurationsdatei: {\texttt /etc/libvirt/qemu/storage/vg0.xml}
\begin{lstlisting}
<pool type='logical'>
        <name>vg0</name>
        <source>
                <device path='/dev/sda1'/>
        </source>
        <target>
                <path>/dev/vg0</path>
        </target>
</pool>
\end{lstlisting}
\begin{lstlisting}[numbers=none]
> virsh pool-define /etc/libvirt/qemu/storage/vg0.xml
> virsh pool-start vg0
> virsh pool-autostart vg0
\end{lstlisting}

\subsubsection{Netzwerk}

Der Hypervisor wurde mit zwei Interfaces an den Switch angebunden. Das Interface {\texttt enp4s0} wurde an einen Port mit VLAN 10 angeschlossen und {\texttt enp2s0} an einen Port mit VLAN 20. Dadurch ist es später einfacher, die virtuellen Server einem VLAN zuzuordnen (s. Kapitel \ql Linux-Server\qr). Für DNS wurde ein Server von Google ausgewählt.\newline

Konfigurationsdatei: {\texttt /etc/network/interfaces}
\begin{lstlisting}
source /etc/network/interfaces.d/*

# The loopback network interface
auto lo
iface lo inet loopback

# The primary network interface
auto enp4s0
iface enp4s0 inet manual
        dns-nameservers 2001:4860:4860::8888

auto enp2s0
iface enp2s0 inet manual
        dns-nameservers 2001:4860:4860::8888

auto br0
iface br0 inet manual

iface br0 inet6 static
        bridge_ports    enp4s0
        address 2001:4dd0:fc0b:a::3
        netmask 64
        gateway 2001:4dd0:fc0b:a::1

auto br1
iface br1 inet manual

iface br1 inet6 static
        bridge_ports    enp2s0
        address 2001:4dd0:fc0b:f4::3
        netmask 64
\end{lstlisting}

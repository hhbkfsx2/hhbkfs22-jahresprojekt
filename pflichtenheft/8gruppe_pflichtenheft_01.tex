\section{Allgemeine Anforderungsbeschreibung}

\subsection{Ausgangssituation}

Die FastForward GmbH nutzt momentan einen IPv4-only-Anschluss und bekommt von ihrem Provider derzeit noch keinen IPv6-Anschluss. Da der IPv4-Adressraum im asiatischen Raum historisch bedingt nicht so groß ist wie beispielsweise der amerikanische, besteht gerade im asiatischen Raum ein erhöhter Bedarf an der zeitnahen Einführung von IPv6. Um die Kommunikation mit Kunden und Zulieferern in Fernost zu optimieren, die bereits IPv6 nutzen soll die eigene Nutzung von IPv6 getestet werden.

\subsection{Zielsetzung}

Es soll zunächst eine Testumgebung erstellt werden, die intern nur auf IPv6 basiert. Die Verbindung in das öffentliche Internet wird über einen Tunnel-Broker realisiert. In dieser Testumgebung sollen ein Web- und ein Mailserver betrieben werden, die öffentlich erreichbar sind. Optional ist darüber hinaus die Installation eines Active Directory und eines Domain Controllers.
\section{Produkteinsatz}

\subsection{Anwendungsbereiche}
Der Anwendungsbereich unserer Lösung umfasst die gesamte Kommunikation, die innerhalb und aus der Testumgebung heraus über die Internetverbindung geführt wird. Die Testumgebung ist ausschließlich eine Machbarkeitsstudie, um auf den tatsächlichen Umstieg zu IPv6 vorbereitet zu sein.

\subsection{Zielgruppe}
Die Zielgruppe dieser Machbarkeitsstudie ist die IT-Abteilung des Unternehmens. Dabei soll ausgelotet werden, ob eine Umstellung des eigenen Netzwerkes auf IPv6 machbar ist. Damit ist die konkrete Zielgruppe das Projektteam, welches später die Umstellung durchführen soll.

\subsection{Betriebsbedingungen}

Die Umgebung soll 24 Stunden 7 Tage die Woche betrieben werden. Eine Beobachtung respektive Monitoring der Systeme ist kein Kriterium und findet entsprechend nur während den Arbeitszeiten an den Projekttagen statt.\\

\noindent{\bf Hardware}: Es stehen ein Router (Cisco 2800 Series), ein Switch (Cisco Catalyst 2960G Series) und ein no-name Server zur Verfügung. Als Test-Clients werden die eigenen Notebooks verwendet.\\
{\bf Orgware}: Es muss einen Aktiven Account bei SixXT, eine IPv4 Adresse des Providers und einen MX Record zur Verfügung gestellt werden.

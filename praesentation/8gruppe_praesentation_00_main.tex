\documentclass{beamer}

\usepackage{graphicx} % Allows including images
\usepackage{booktabs} % Allows the use of \toprule, \midrule and \bottomrule in tables
\usepackage[utf8]{inputenc}

\usepackage{/home/kayle/latex/keynote-gradient}

%----------------------------------------------------------------------------------------
%	TITLE PAGE
%----------------------------------------------------------------------------------------

\title[Jahresprojekt FS22 - Gruppe 8]{Umsetung eines IPv6-Netzwerksegmentes mit Internetanbindung für die FastForward GmbH} % The short title appears at the bottom of every slide, the full title is only on the title page

\author{Kilian Engelhardt, Mirko Großmann, Tom Vogler} % Your name
%\institute[] % Your institution as it will appear on the bottom of every slide, may be shorthand to save space
%{
%\medskip
%\textit{kilian.engelhardt@gmail.com} % Your email address
%}
%\date{\today} % Date, can be changed to a custom date

\begin{document}

\begin{frame}
\titlepage % Print the title page as the first slide
\end{frame}

\begin{frame}
\frametitle{Struktur} % Table of contents slide, comment this block out to remove it
\tableofcontents % Throughout your presentation, if you choose to use \section{} and \subsection{} commands, these will automatically be printed on this slide as an overview of your presentation
\end{frame}

%----------------------------------------------------------------------------------------
%	PRESENTATION SLIDES
%----------------------------------------------------------------------------------------

%------------------------------------------------
\section{Informationen}
%------------------------------------------------
\begin{frame}[label=informationen]
\frametitle{Informationen}
\end{frame}

\begin{frame}
\frametitle{}

\includegraphics[width=\textwidth,height=\textheight,keepaspectratio]{pk_bong_betriebsrat_anzahl2.png}

\end{frame}
%------------------------------------------------
\section{Router}
%------------------------------------------------
\begin{frame}[label=router]
\frametitle{Router}
\end{frame}
%------------------------------------------------
\section{Switch}
%------------------------------------------------
\begin{frame}[label=switch]
\frametitle{Switch}
\end{frame}
%------------------------------------------------
\section{Hypervisor}
%------------------------------------------------
\begin{frame}[label=hypervisor]
\frametitle{Hypervisor}
\end{frame}
%------------------------------------------------
\section{Webserver}
%------------------------------------------------
\begin{frame}[label=webserver]
\frametitle{Webserver}
\end{frame}
%------------------------------------------------
\section{Domain Controller}
%------------------------------------------------
\begin{frame}[label=domain-controller]
\frametitle{Domain Controller}
\end{frame}
%------------------------------------------------
\section{Tests}
%------------------------------------------------
\begin{frame}[label=tests]
\frametitle{Tests}
\end{frame}
%----------------------------------------------------------------------------------------
% Fragen und Danke fuer euere Aufmerksamkeit
%----------------------------------------------------------------------------------------
\begin{frame}[label=fragen]
\Huge{\centerline{Fragen?}}
\end{frame}

\begin{frame}[label=danke]
\Huge{\centerline{Danke für eure Aufmerksamkeit}}
\end{frame}

\end{document} 
